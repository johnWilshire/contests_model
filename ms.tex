\documentclass[a4paper,11pt]{article}
\usepackage[osf]{mathpazo}
\usepackage{ms}
\usepackage[]{natbib}
\raggedright

\newcommand{\smurl}[1]{{\footnotesize\url{#1}}}
\usepackage{graphicx}

\title{Toward dynamic models of combat or lack thereof in animal mating}
\author{
* John Wilshire$^1$, Will Cornwell$^1$ , Daniel Falster$^2$, Michael Kasumovic$^1$, \\
Daniel Noble$^1$,$^1$ Loic Thibaut$^1$}
\affiliation{
*final list and order undecided\\
$^1$ University of NSW\\
$^2$ Macquarie University\\
}
\date{}

\bibliographystyle{mee}

\usepackage[title,titletoc,toc]{appendix}

\mstype{Research Article}
\runninghead{A new framework for fighting}
\keywords{}

\begin{document}
\mstitlepage
\noindent
% \doublespacing
% \linenumbers

\section{Summary}
The diversity of animal mating systems is astounding. In some of these
systems, very costly combat behaviour -- among males, among females, or
both -- is a feature of the mating process.  In other systems, resources
are divided among individuals in an entirely pacific process.  Can we
understand why? Animal combat strategies likely emerge from trade-offs
in investment in growth, mate seeking, and information gathering.
Willingness to engage in combat is a trait that evolves based on the
fitness landscape, which itself changes depending on both the
environment and the strategies of other individuals.  Using recently
developed methods for modelling dynamic fitness landscapes, we examine:
(1) why combat behaviours arise, (2) under what conditions combat
behaviours are evolutionarily stable, and (3) when different combat
strategies co-exist.  We hypothesize that the reliability and
"public-ness" of information is an important feature driving combat or
lack thereof in many animal systems.

\section{Introduction}

Evolution of animal personalities: \citep{Wolf-2007,Wolf-2012} show can have
coexistence of risky, explorative strategies and risk-averse strategies.

Animal personalities linked to other life history traits: \citep{Biro-2008}

Individual-based models of natural selection: \citep{MGonigle-2012}


\section{Methods}

% Verbal description of model

We consider a population of males competing for mates. The population is one with non-overlapping generations, and within each generation there is a predefined annual cycle. In spring, up to $K$individuals hatch from eggs, grow throughout spring and summer to increase size, and then within a short period, the entire population mates, females lay eggs and everyone dies. The short duration of mating period is such that each male and female only mates once. The population then re-establishes from eggs the following spring.

The reproductive success of males in the population is determined via their ability to compete for and hold $N$ nests. Nests differ in quality (potential number of offspring per nest per year). The male population is divided into two groups: immature and mature males. Immature males devote all their efforts towards feeding, increasing their size. Males mature with a probability defined by their size and a maturation trait. Once they enter the mature population, males stop feeding and focus solely on obtaining nests. Nests can be vacant or occupied and males are either ``searching'' or ``occupying'' a nest. Males differ in the rate at which they search, and thus the rate at which they encounter nests. Empty nests can be immediately occupied with no additional cost. Filled nests can only be obtained by entering into and winning a contest. Once occupied, males can choose to occupy a nest or continue searching. At the end of each season, each male has a fitness defined by the reproductive value of the nest it holds, $R_i$.

\subsection{Habitat}

Denote $N$ to be the total number of available nests, $R_i$ to be the reproductive value of nest $i$ (number of offspring per nest per year), $\bar{R}$ to be the average reproductive value of all nests (number of offspring per nest per year), and $p(R)$ to be the probability density distribution of nest quality with respect to $R$. The distribution of nest qualities is then set according to the distribution
\begin{equation} \label{eq:pdf_R}
    Pr(R_i = x) =\bar{R} \, p(R).
\end{equation}
The total number of male offspring produced in each generation is then
\begin{equation} \label{eq:pdf_R}
    K = \bar{R} N.
\end{equation}

\subsection{Immature phase}

Within the immature phase, we assume males grow with rate
\begin{equation} \label{eq:growth}
\frac{{\rm d} m}{{\rm d} t} = a \, m^ b
\end{equation}
where $a$ is mass-based rate of growth. Integrating eq. \ref{eq:growth}, the size of individuals at time $t$, having started growing at time $0$ is given by
\begin{equation} \label{eq:growth}
m(t) = \left(m(0)^{1-b} + t a(1-b)\right)^{\frac1{1-b}},
\end{equation}
while the time $\tau$ taken for a male to grow from $m(0) \rightarrow m$ is then
\begin{equation} \label{eq:tau}
\tau(m_0, m) = \frac1{a(1-b)}\left(m ^{1-b} - m_0 ^{1-b}\right).
\end{equation}

If we assume a constant mortality rate $k_i$ during the immature phase, the probability of males surviving to time $t$ is given by
\begin{equation} \label{eq:surv_immature}
S(t) = \exp(-k_i \, t).
\end{equation}

Finally, we assume males transition from immature $\rightarrow$ mature according to logistic probability function
\begin{equation}\label{eq:allocation}
Pr({\rm mature} \, | m) = \frac1{1 + \exp\left(\alpha \left(1 - m / m_{\rm mat}\right)\right)},
\end{equation}
where $m_{\rm mat}$ is the size at which 50\% of the population is mature and $\alpha$ determines the width of the curve.

\subsection{Mature phase}
When a male matures it is given an energy budget linearly proportional to its mass, this is a parameter into the model.\\


\subsection{Searching Males}
When males mature they begin searching for a females nest


The area that a male $m$ with speed trait $v$ and radius trait $r$ can search in time step $dt$ is
\begin{equation}
    A_{searched} = 2r \cdot v \cdot dt
\end{equation}
See Fig. \ref{fig:area searched}.

We let the total area enclosing all males and all nests be a parameter, $A_{total}$.

The probability of a male that covers $A_{searched}$ area in time period $dt$ encountering any nest is:
\begin{equation}
    P(\rm {discovery}) = \frac {N \cdot A_{searched}} {A_{total}}
\end{equation}
\citep{Gillespie-1976}
\\
TODO metabolic cost of searching with r and v ?\\

\subsection{Occupying males}
The probability of a male abandoning a nest $j$, is a function of the  the reproductive value of the nest $R_j$ and the abandon trait of the individual $\psi_i$: 

assumption: $1 < R_j$,  $1 < \psi_i$

\begin{equation} \label{eq:abandon}
    Pr({\rm abandon} | R_j ) = \frac{1}{1 + \psi_i \cdot R_j}
\end{equation}


\subsection{Evolutionary dynamics}

A new individual will inherit the fathers genes,
There is a global mutation rate parameter which is the chance that a trait will not be inherited perfectly but instead will be mutated.
This means adding some normally distributed noise with $\mu = 0$ and $\sigma$ varying from trait to trait.

\begin{enumerate}
    \item Abandon Rate, influences RR abandonment rate (not yet implemented)
    Trade offs:
    \begin{itemize}
        \item more likely to abandon a nest and not discover another before the females mature
    \end{itemize}

    \item Aggressiveness, influences escalation in contests
    Trade offs:
    \begin{itemize}
        \item more likely to engage in fights with a low probability of winning
    \end{itemize}
    
    \item search radius, the radius of a searching male (can be thought of as line of sight), See Fig. \ref{fig:area searched}.
    Trade offs:
    \begin{itemize}
        \item higher metabolic cost of search
    \end{itemize}

    \item search speed, the distance that a searching male will cover in one time step, see Fig. \ref{fig:area searched}.
    Trade offs:
        \begin{itemize}
            \item higher metabolic cost of search
            \item higher predation rate
        \end{itemize}
\end{enumerate}

\subsection{Algorithms}
\subsection{Event Algorithms}
So far we are considering two algorithms for stepping through events one is a stochastic simulation algorithm and a simpler (although more compuationally expensive) stepping algorithm.
\subsection{SSA}
Simulating the population dynamics requires accounting for four types of events:

An efficient method for stepping the population is to step between successive events. So rather than stepping the system over a given time interval and considering any transitions that may occur in that time period, the system is stepped between successive events by drawing the next event and the time at which the event occurs from appropriate distributions.  We can identify time to next event using Gillespie minimal process method \citep{Gillespie-1976}, which indicates that with a given event rate, the distribution of waiting times follows an exponential distribution.


In the SSA algorithm \citep{Gillespie-1976} there are 6 different states
\begin{enumerate}
    \item $S_1 = $ immature male
    \item $S_2 = $ mature searching male
    \item $S_3 = $ unoccupied nest
    \item $S_4 = $ occupied nest
    \item $S_5 = $ recently defeated male
    \item $S_6 = $ recently changed nest
\end{enumerate}

these states have $R_\mu$ different interactions
\begin{table}[h!]
    \caption{state changes}
    \centering
    \begin{tabular}{c | c | c | c | l }
        \hline
            Event Number & reactants  & & products & Description\\
        \hline
        \hline
            $R_1$       &   $S_1$       & $--c_1-->$      &   $ S_2$      & maturation\\
            $R_2$       &   $S_2 + S_3$ & $--c_2-->$      &   $S_4 $      & occupation\\
            $R_3$       &   $S_2 + S_4$ & $--c_3-->$      &   $S_4 + S_5$ & contest\\
            $R_4$       &   $S_5$       & $--c_4-->$      &   $S_2$       & recovery\\
            $R_5$       &   $S_2 + S_4$ & $--c_5-->$      &   $S_4$       & contest with death\\
            $R_6$       &   $S_4$       & $--c_6-->$      &   $S_2 + S_3$ & abandon event\\
            $R_7$       &   $S_1$       & $--c_7-->$      &   $0$         & metabolic / predation event\\
            $R_8$       &   $S_2$       & $--c_8-->$      &   $0$         & predation event\\
            $R_9$       &   $S_4$       & $--c_9-->$      &   $S_3$       & metabolic / predation event\\
            $R_{10}$    &   $S_6$       & $--c_{10}-->$   &   $S_4$       & nest handover\\
        \hline
    \end{tabular}

\end{table}

$c_n$ is called the stochastic reaction constant\\
$c_n \cdot dt = $ the average probability that a particular combination of $R_n$, reactant molecules will react accordingly in the next infinitesimal time interval $dt$\\


\citep{Gillespie-1976}


\clearpage
\subsection{Naive Event Algorithm}
we have 3 lists
\begin{verbatim}
    immature males
    searching males
    nests
\end{verbatim}

\begin{verbatim}
time = 0
while time < time when the females mature:
    check which males have matured
        add them to the list of searching males
    for each occupied nest:
        deduct metabolic costs from the occupier
        remove dead males

    for each searching male m:
        deduct metabolic costs from m
        remove dead males
        reduce time until next event for m
        if time until next event <= 0:
            reset the time until next event
            if random > exploration_prob:
                choose a nest at random
                if it is occupied:
                    contest the occupying male for control of the nest
                    if m has won the contest:
                        m occupy's n
                        the loser is added to the list of searching males
                else:
                    occupy the nest
    t += delta_time
\end{verbatim}


\clearpage
\subsection{Contest Algorithm}
See Fig. \ref{fig:contests}.
For a defender $x_{def}$ vs an attacker $x_{atk}$, there are 3 (currently 2) stages:
each individual has an aggression trait $a$ and a mass $m$.
\begin{equation}\label{eq:logit}
    logit^{-1}(\alpha) = \frac{exp(\alpha)}{1 + exp(\alpha)}
\end{equation}

$m$ is mass
$a$ is aggression trait
\begin{enumerate}
    \item Display 1
        Outcomes:
        \begin{itemize}
            \item $x_{atk}$ escalates: 
                $Pr(\rm{escalation}) =  logit^{-1}(a_{atk} \cdot m_{atk} - m_{def})$
                or leaves.
            \item $x_{def}$ escalates:
                $Pr(\rm{escalation}) =  logit^{-1}(a_{def} \cdot m_{def} - m_{atk})$ 
                or leaves.
            \item if the attacker leaves the defender keeps the nest
            \item if the defender leaves the attacker keeps the nest
            \item the attacker is evaluated first
            \item if they both escalate then the next stage occurs
        \end{itemize}
    \item Display 2:
        similar to stage 1, this could involve the $R$ value of the nest, or include the oponents aggression levels or energy reseves(?).
    \item Fight
        \begin{itemize}
            \item $x_{def}$ wins:
            $Pr(\rm{def}) = logit^{-1}(m_{def} - m_{atk})$
        \end{itemize}
\end{enumerate}
The cost of the fighting stage is much greater than the cost of the display stage.

After the contest Energy costs are deducted from both the winner and the loser.

The winner also has a chance to abandon the RR (based off a trait).\\

If a death occurs (Energy(x) less than 0) then x is removed from the individual pool.


\clearpage

I am not sure if this is the best way to do this, the tables in the plant paper are more aesthetic.\\

\begin{table}[h!]
    \caption{Variable names and definitions in the model.}
    \centering
    \begin{tabular}{c | c | l }
        \hline
        Symbol & unit & Description\\
        \hline
        \hline
        $K$ & & number of individual males \\
        $N$ & & number of nests \\
        $R_i$ & & repoductive value of nest of nest $i$\\
        $\bar{R}$ & & average repoductive value of nest\\
        \hline
        $t_m$ & & the time at which the females begin to mature\\
        $S(t)$ = exp($-k_it$) & & The probability of an individual surviving to time t.\\
        \hline
        $m$ & $g$ & mass\\
        $m_mat$ & $g$ & The size at which 50\% of the population is mature.\\
        $E$ & kJ & kiloJoules\\
        \hline
    \end{tabular}
\end{table}

\section{Figures}

\begin{figure}[h!]
\centering
\includegraphics[width=10cm,height=10cm,keepaspectratio]{figures/event_algorithm}
\caption{Algorithm for modelling dynamics within the ``mature'' male population.}
\label{fig:events}
\end{figure}
\clearpage

\begin{figure}[h!]
\centering
\includegraphics[width=15cm,height=15cm,keepaspectratio]{figures/contest_algorithm}
\caption{Algorithm for modelling contests behaviour within the ``mature'' male population.}
\label{fig:contests}
\end{figure}
\clearpage

\begin{figure}[h!]
\centering
\includegraphics[width=15cm,height=15cm,keepaspectratio]{figures/area_searched}
\caption{the area searched my a male in interval $dt$}
\label{fig:area searched}
\end{figure}
\clearpage

\bibliography{refs}

\end{document}

