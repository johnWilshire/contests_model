\documentclass[a4paper,11pt]{article}
\usepackage[osf]{mathpazo}
\usepackage{ms}
\usepackage[]{natbib}
\raggedright

\newcommand{\smurl}[1]{{\footnotesize\url{#1}}}



\newcommand{\myfigure}[4]{
\begin{figure}[h!]
    \centering
    \includegraphics[width=13cm,height=10cm]{#1}
    \caption{#2}
    \centering
    \includegraphics[width=13cm,height=10cm]{#3}
    \caption{#4}
\end{figure}
}





\usepackage{graphicx}

\title{Toward dynamic models of combat or lack thereof in animal mating}
\author{
* John Wilshire$^1$, Will Cornwell$^1$ , Daniel Falster$^2$, Michael Kasumovic$^1$, \\
Daniel Noble$^1$,$^1$ Loic Thibaut$^1$}
\affiliation{
*final list and order undecided\\
$^1$ University of NSW\\
$^2$ Macquarie University\\
}
\date{}

\bibliographystyle{mee}

\usepackage[title,titletoc,toc]{appendix}

\mstype{Research Article}
\runninghead{A new framework for fighting}
\keywords{}

\begin{document}
\mstitlepage
\noindent
% \doublespacing
% \linenumbers

\section{Summary}
The diversity of animal mating systems is astounding. In some of these
systems, very costly combat behaviour -- among males, among females, or
both -- is a feature of the mating process.  In other systems, resources
are divided among individuals in an entirely pacific process.  Can we
understand why? Animal combat strategies likely emerge from trade-offs
in investment in growth, mate seeking, and information gathering.
Willingness to engage in combat is a trait that evolves based on the
fitness landscape, which itself changes depending on both the
environment and the strategies of other individuals.  Using recently
developed methods for modelling dynamic fitness landscapes, we examine:
(1) why combat behaviours arise, (2) under what conditions combat
behaviours are evolutionarily stable, and (3) when different combat
strategies co-exist.  We hypothesize that the reliability and
"public-ness" of information is an important feature driving combat or
lack thereof in many animal systems.

\section{Introduction}

Evolution of animal personalities: \citep{Wolf-2007,Wolf-2012} show can have
coexistence of risky, explorative strategies and risk-averse strategies.

Animal personalities linked to other life history traits: \citep{Biro-2008}

Individual-based models of natural selection: \citep{MGonigle-2012}


\section{Methods}

% Verbal description of model

We consider a population of males competing for mates. The population is one with non-overlapping generations, and within each generation there is a predefined annual cycle. In spring, up to $K$ individuals hatch from eggs, grow throughout spring and summer to increase size, and then within a short period, the entire population mates, females lay eggs and everyone dies. The short duration of mating period is such that each male and female only mates once. The population then re-establishes from eggs the following spring.

The reproductive success of males in the population is determined via their ability to compete for and hold $N$ nests. Nests differ in quality (potential number of offspring per nest per year). The male population is divided into two groups: immature and mature males. Immature males devote all their efforts towards feeding, increasing their size. Males mature with a probability defined by their size and a maturation trait. Once they enter the mature population, males stop feeding and focus solely on obtaining nests. Nests can be vacant or occupied and males are either ``searching'' or ``occupying'' a nest. Males differ in the rate at which they search, and thus the rate at which they encounter nests. Empty nests can be immediately occupied with no additional cost. Filled nests can only be obtained by entering into and winning a contest. At the end of each season, each male has a fitness defined by the fecundity of the nest it holds.

\subsection{Habitat}

Denote $N$ to be the total number of available nests, The fecundity (number of offspring per nest per year) is uniform across the population of nests (females) and is given the symbol $F$.

The total number of male offspring produced in each generation is then
\begin{equation} \label{eq:pdf_F}
    K = F N
\end{equation}
The nests are randomly distributed.

We use $F = 20$ and $N = 500$

\subsection{Immature phase}

Within the immature phase, we assume males grow with rate
\begin{equation} \label{eq:growth}
    \frac{dm}{dt} = a m^ b
\end{equation}

where $a$ is mass-based rate of growth. Integrating eq. \ref{eq:growth}, the size of individuals at time $t$, having started growing at time $0$ is given by
\begin{equation} \label{eq:growth}
    m(t) = \left(m(0)^{1-b} + t a(1-b)\right)^{\frac1{1-b}},
\end{equation}

We use $a = 0.5$ and $b = 0.1$ and $m(0) = 5$


We Pull the maturation time of a male randomly from the logistic distribution with center $= 4$ and width $= 1$.

If we assume a constant mortality rate $k_{im}$ during the immature phase, the probability of a male surviving to time $t$ from $0$ is given by
\begin{equation} \label{eq:surv_immature}
    S(t) = \exp(-k_{im} t).
\end{equation}

We use $k_{im} = 0.2$

For example the probability of a male surviving to maturity from to time = $5$ is $exp(-0.2 \cdot 5)= 0.3678$.


When a male matures it is given an energy budget linearly proportional to its mass such that: 
$$ E = k_em$$

We use $k_e = 10$

After a male matures it cannot gain more energy. When a mature male uses its energy allocation it dies.



\subsection{Mature phase}
\subsection{Searching Males}
When males mature they begin searching for a females nest.


The area that a male with speed trait $v$ and radius trait $r$ can search in one time step is
\begin{equation}
    2 \cdot r \cdot v
\end{equation}
A male is 
See Fig. \ref{fig:area searched}.

The energy used by a searching male with speed $v$ and radius $r$ is 
\begin{equation}
    E(r,v) = k_{search} \cdot r \cdot v
\end{equation}
where $k_{search} = 5$ is a constant.  % state value

We let the total area enclosing all nests be a parameter, $A_{total}$.
$N$ is the number of nests enclosed. $\frac{N}{A_{total}}$ is the nest density.

The encounter rate $\lambda$ for an individual is:
\begin{equation} \label{eq:encounter rate}
    \lambda = \frac {N \cdot 2 \cdot r \cdot v} {A_{total}}
\end{equation}
\citep{Gurarie2012}

We assume spatially random nests so nest discovery is a poisson process with the the expected number of encounters in one time step being $\lambda$.
Because of this we can then use $\lambda$ as the rate parameter to repeatedly draw the time of the next event for this individual from the exponential distribution.

We do this for each male from when they mature up to when the females mature (at $\text{time} = 10$).
When an event occurs we select a nest randomly and then the male will either occupy the nest or contest another male for control of the nest.

When a male loses a contest it is kicked out of the nest, we discard the old events and generate a list of new ones.

\clearpage

\subsection{Contests}
When a searching male encounters a nest that is occupied a contest over possession of that nest occurs.

Contests occur instantaneously, taking up no time, this means that another male cannot encounter the nest whilst the contest is taking place, additionally the contest has no intrinsic metabolic cost.

At the start of the contest each male chooses how much  energy they are willing to commit to this contest, they have full information about their opponents mass.

They do this by reporting their level of commitment as a positive value $c$.

The intensity of a fight increases with more commitment:

We define $\text{Intensity} = \frac{\text{d}Energy}{\text{d}Commitment}$

We use an intensity curve $I(c) = c^2$

\begin{figure}[h!]
    \centering
    \includegraphics[width=10cm,height=8cm]{figures/example_escalation.png}
    \caption{The energy spent in the fight between male i and male j is the pink shaded area.}
\end{figure}

To get the cost of the fight we take the integral of the intensity function:
$$E_\text{cost}(c) = \int_0^{c}{I(t)}dt = \int_0^{c}{t^2}dt $$

The lower energy cost ( corrosponding to $c_i$ in the diagram) will then be deducted from both males.

Male $i$ will loose the contest with probability 
$$P(\text{})= logit^{-1}(k(c_j - c_i))$$

Where k is is a parameter that shapes how sensitive the contest is to small differences in commitment values (we use $k = 10$).
can lower k be thought of as perception error (?? mike).

The male chooses how much energy to commit based off of a commitment function. For a male $i$ against a male $j$

$$\text{commitment} = e^\beta \cdot (\frac{m_i}{m_j})^\alpha $$

Where the trait values:
$$\alpha , \beta$$
are under selection.

\clearpage

\subsection{Evolutionary dynamics}

A new individual will inherit the fathers genes,
There is a mutation rate parameter which is the chance that a trait will not be inherited perfectly but instead will be mutated.

This means a chance to add some normally distributed noise with $\mu = 0$ and small $\sigma$.


\begin{enumerate}
    \item $\beta$
    \item $\alpha$
    \item $r$, the search radius of a searching male, See Fig. \ref{fig:area searched}.
    \item $v$, the search speed of a male.
\end{enumerate}

The initial distribution of $\alpha$ and $\beta$ is $N(0, 3)$.

The initial distribution of $v$ and $r$ is $N(2, 0.5)$.

Where $N(\mu, \sigma)$, is the normal distribution with mean $= \mu$ and standard deviation $\sigma$.

\section{Results} % (fold)
\label{sec:results}

%\myfigure{figures/energy_pc_plot_fixed_r_v.png}{pc energy spent on what}{figures/energy_total_plot_fixed_r_v.png}{total energy spent on what, fixed r_v}

\begin{figure}[h!]
    \centering
    \includegraphics[width=15cm,height=10cm]{figures/clustered_e_0_plot.pdf}
    \caption{A plot of the distribution of the aggression parameter $beta$ over varing $A_{total}$ with k means clustering done, with the k parameter chosen by $pamk$, the number of clusters (coexisting strategies) can be be seen on the right, r - v fixed}
\end{figure}



\clearpage
% section results (end)results
\begin{table}[h!]
    \caption{Variable names and definitions in the model.}
    \centering
    \begin{tabular}{c | l }
        \hline
        Symbol & Description\\
        \hline
        \hline
        $N$ &  The number of nests \\
        $R$ &  The reproductive value of nest of a nest\\
        $t_m$  & The time at which the females to mature\\
        $A_{total}$ & The area which encloses the nests\\
        \hline
        $k_{im}$ & Immature mortality coefficient \\
        $k_e$ & The initial mass to energy coefficient \\
        $k_{search}$ & The metabolic cost of searching \\
        \hline
        Individual & \\
        \hline
        $m$ & mass\\
        $E$ & The current energy of an individual\\
        $r$ & An individuals search radius\\
        $v$ & An individuals speed\\ 
        \hline
        mutation rate & The chance of a trait being mutated\\
        mutation sd & The standard deviation of  these mutations\\
        \hline
    \end{tabular}
\end{table}

\bibliography{refs}

\end{document}
