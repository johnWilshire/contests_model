\documentclass[a4paper,11pt]{article}
\usepackage[osf]{mathpazo}
\usepackage{ms}
\usepackage[]{natbib}
\raggedright

\newcommand{\smurl}[1]{{\footnotesize\url{#1}}}
\usepackage{graphicx}

\title{Toward dynamic models of combat or lack thereof in animal mating}
\author{
* John Wilshire$^1$, Will Cornwell$^1$ , Daniel Falster$^2$, Michael Kasumovic$^1$, \\
Daniel Noble$^1$,$^1$ Loic Thibaut$^1$}
\affiliation{
*final list and order undecided\\
$^1$ University of NSW\\
$^2$ Macquarie University\\
}
\date{}

\bibliographystyle{mee}

\usepackage[title,titletoc,toc]{appendix}

\mstype{Research Article}
\runninghead{A new framework for fighting}
\keywords{}

\begin{document}
\mstitlepage
\noindent
% \doublespacing
% \linenumbers

\section{Summary}
The diversity of animal mating systems is astounding. In some of these
systems, very costly combat behaviour -- among males, among females, or
both -- is a feature of the mating process.  In other systems, resources
are divided among individuals in an entirely pacific process.  Can we
understand why? Animal combat strategies likely emerge from trade-offs
in investment in growth, mate seeking, and information gathering.
Willingness to engage in combat is a trait that evolves based on the
fitness landscape, which itself changes depending on both the
environment and the strategies of other individuals.  Using recently
developed methods for modelling dynamic fitness landscapes, we examine:
(1) why combat behaviours arise, (2) under what conditions combat
behaviours are evolutionarily stable, and (3) when different combat
strategies co-exist.  We hypothesize that the reliability and
"public-ness" of information is an important feature driving combat or
lack thereof in many animal systems.

\section{Introduction}

Evolution of animal personalities: \citep{Wolf-2007,Wolf-2012} show can have
coexistence of risky, explorative strategies and risk-averse strategies.

Animal personalities linked to other life history traits: \citep{Biro-2008}

Individual-based models of natural selection: \citep{MGonigle-2012}


\section{Methods}

% Verbal description of model

We consider a population of males competing for mates. The population is one with non-overlapping generations, and within each generation there is a predefined annual cycle. In spring, up to $K$ individuals hatch from eggs, grow throughout spring and summer to increase size, and then within a short period, the entire population mates, females lay eggs and everyone dies. The short duration of mating period is such that each male and female only mates once. The population then re-establishes from eggs the following spring.

The reproductive success of males in the population is determined via their ability to compete for and hold $N$ nests. Nests differ in quality (potential number of offspring per nest per year). The male population is divided into two groups: immature and mature males. Immature males devote all their efforts towards feeding, increasing their size. Males mature with a probability defined by their size and a maturation trait. Once they enter the mature population, males stop feeding and focus solely on obtaining nests. Nests can be vacant or occupied and males are either ``searching'' or ``occupying'' a nest. Males differ in the rate at which they search, and thus the rate at which they encounter nests. Empty nests can be immediately occupied with no additional cost. Filled nests can only be obtained by entering into and winning a contest. Once occupied, males can choose to occupy a nest or continue searching. At the end of each season, each male has a fitness defined by the reproductive value of the nest it holds, $R_i$.

\subsection{Habitat}

Denote $N$ to be the total number of available nests, $R_i$ to be the reproductive value of nest $i$ (number of offspring per nest per year), $\bar{R}$ to be the average reproductive value of all nests (number of offspring per nest per year), and $p(R)$ to be the probability density distribution of nest quality with respect to $R$. The distribution of nest qualities is then set according to the distribution
\begin{equation} \label{eq:pdf_R}
    Pr(R_i = x) =\bar{R} \, p(R).
\end{equation}
The total number of male offspring produced in each generation is then
\begin{equation} \label{eq:pdf_R}
    K = \bar{R} N.
\end{equation}

\subsection{Immature phase}
% TODO update this to reflect what happens in the model

Within the immature phase, we assume males grow with rate
\begin{equation} \label{eq:growth}
\frac{{\mathrm d} m}{{\mathrm d} t} = a \, m^ b
\end{equation}
where $a$ is mass-based rate of growth. Integrating eq. \ref{eq:growth}, the size of individuals at time $t$, having started growing at time $0$ is given by
\begin{equation} \label{eq:growth}
m(t) = \left(m(0)^{1-b} + t a(1-b)\right)^{\frac1{1-b}},
\end{equation}

We Pull the maturation time from the logistic distribution with a  location and width parameters.

If we assume a constant mortality rate $k_i$ during the immature phase, the probability of males surviving to time $t$ is given by
\begin{equation} \label{eq:surv_immature}
S(t) = \exp(-k_i \, t).
\end{equation}

Finally, we assume males transition from immature $\rightarrow$ mature according to logistic probability function
\begin{equation}\label{eq:allocation}
Pr({\mathrm mature} \, | m) = \frac1{1 + \exp\left(\alpha \left(1 - m / m_{\mathrm mat}\right)\right)},
\end{equation}
where $m_{\mathrm mat}$ is the size at which 50\% of the population is mature and $\alpha$ determines the width of the curve.

\subsection{Mature phase}
When a male matures it is given an energy budget linearly proportional to its mass, this is a parameter into the model.\\


\subsection{Searching Males}
When males mature they begin searching for a females nest.

The area that a male $m$ with speed trait $v$ and radius trait $r$ can search in one time step is
\begin{equation}
    2 \cdot r \cdot v
\end{equation}
See Fig. \ref{fig:area searched}.

We let the total area enclosing all males and all nests be a parameter, $A_{total}$. We assume spatially random nests.

The encounter rate $\varepsilon$ for an individual is:
\begin{equation} \label{eq:encounter rate}
    \varepsilon = \frac {N \cdot 2 \cdot r \cdot v} {A_{total}}
\end{equation}
\citep{Gurarie2012}

We then use this as the rate parameter $lambda$ to repeatedly draw the time of the next discovery event for this individual from the exponential distribution, we do this for each male from when they mature up to when the females mature.
When an event occurs we select a nest randomly and then the male will either occupy it or contest another male for it.

When a male is kicked out of a nest we discard the old events and generate new ones.

The energy used by a male with speed $v$ and radius $r$ is 
\begin{equation}
    E(r,v) = k_{search} \cdot r \cdot v
\end{equation}
where $k_{search}$ is a constant.


\clearpage
\subsection{Contest Algorithm}
Contests 
When a searching male encounters a nest that is occupied a contest over possession of that nest occurs. Contests occur instantaneously. 

The costs of a contest escalates with more commitment.

We will assume an escalation curve $kappa$ such that:
$$\kappa : Commitment \rightarrow Intensity$$
where $Intensity$ is $\frac{Energy}{Commitment}$

We use the escalation curve $$ \kappa (c) = c^2 $$

At the start of the contest each male chooses how much  energy they are willing to commit.
They do this by reporting their level of ``commitment'' $c_i$.

The male that chooses to commit more of its energy to the contest will win with probability $= logit^(k(c_i - c_j))$. Where k is is a parameter (I reduced it to 10 (also what )). 

The energy that this male chose to commit will then be deducted from both males with the loser continuing to search for mates

To get the cost of the fight we take the integral of $\kappa$:

$$Energy Spent = \int_0^{c_i}{\kappa(t)}dt$$

if the $\kappa$ is $commitment^2$ then
$$EnergySpent = \frac{c_i^3}{3}$$
Where $c_i$ is the losing male.


The male chooses how much energy to commit based off of a commitment function, we examine two different commitment functions, for a male $i$ against a male $j$

An exponential one:
$$\mathrm{commitment} = e^\beta \cdot (\frac{m_i}{m_j})^\alpha $$

And a linear one:

$$\mathrm{commitment} = \beta + \alpha \cdot (\frac{m_i - m_j}{m_i + m_j})$$

Where the trait values:
$$\alpha , \beta$$
Are under selection.


\clearpage

\subsection{Evolutionary dynamics}

A new individual will inherit the fathers genes,
There is a global mutation rate parameter which is the chance that a trait will not be inherited perfectly but instead will be mutated.
This means adding some normally distributed noise with $\mu = 0$ and $\sigma$ varying from trait to trait.

\begin{enumerate}
    \item Abandon Rate, influences RR abandonment rate (not yet implemented)
    Trade offs:
    \begin{itemize}
        \item more likely to abandon a nest and not discover another before the females mature
    \end{itemize}

    \item aggression 0 ($\beta$) called $e_0$
    \item aggression 1 ($\alpha$) called $k$
    
    \item Search radius, the radius of a searching male (can be thought of as line of sight), See Fig. \ref{fig:area searched}.
    Trade offs:
    \begin{itemize}
        \item higher metabolic cost of search
    \end{itemize}

    \item search speed, the distance that a searching male will cover in one time step, see Fig. \ref{fig:area searched}.
    Trade offs:
        \begin{itemize}
            \item higher metabolic cost of search
            \item higher predation rate
        \end{itemize}
    \item maturation time, the time at which to mature is inherited as well but with a constant level of noise
\end{enumerate}

\clearpage

\section{Results} % (fold)
\label{sec:results}
% energy expenditure plots
\begin{figure}[h!]
    \centering
    \includegraphics[width=10cm,height=10cm,keepaspectratio]{figures/energy_pc_plot_exp_f10.pdf}
    \caption{Exponential commitment function, ftime = 10. Percent energy expenditure of cohort at generation 499}
    \label{fig:energy pc}

    \centering
    \includegraphics[width=10cm,height=10cm,keepaspectratio]{figures/energy_total_plot_exp_f10.pdf}
    \caption{Exponential commitment function, ftime = 10. Total energy expenditure of cohort at generation 499}
    \label{fig:energy total}
\end{figure}
\clearpage

\begin{figure}[h!]
    \centering
    \includegraphics[width=10cm,height=10cm,keepaspectratio]{figures/energy_pc_plot_exp_f15.pdf}
    \caption{Exponential commitment function, ftime = 15. Percent energy expenditure of cohort at generation 499}
    \label{fig:var energy pc}
    \centering

    \includegraphics[width=10cm,height=10cm,keepaspectratio]{figures/energy_total_plot_exp_f15.pdf}
    \caption{Exponential commitment function, ftime = 15. Total energy expenditure of cohort at generation 499}
    \label{fig:var energy total}
\end{figure}

% parameter plots density  plots

% exp 10
\begin{figure}[h!]
    \centering
    \includegraphics[width=10cm,height=10cm,keepaspectratio]{figures/exp_f10_k_density.pdf}
    \caption{Exponential commitment function, ftime = 10. a density plot of the aggression parameter k}
    \label{fig:k density}

    \includegraphics[width=10cm,height=10cm,keepaspectratio]{figures/exp_f10_e_0_density.pdf}
    \caption{Exponential commitment function, ftime = 10. a density plot of the aggression parameter $e_0$}
    \label{fig:e_0 density}
\end{figure}

\begin{figure}[h!]
    \centering
    \includegraphics[width=10cm,height=10cm,keepaspectratio]{figures/exp_f10_k_history.pdf}
    \caption{Exponential commitment function, ftime = 10. $k$ the grey bounds are the standard deviation}
    \label{fig:k history}

    \includegraphics[width=10cm,height=10cm,keepaspectratio]{figures/exp_f10_e_0_history.pdf}
    \caption{ Exponential commitment function, ftime = 10. $e_0$ the grey bounds are the standard deviation}
    \label{fig:e_0 history}
\end{figure}
% exp 15

\begin{figure}[h!]
    \centering
    \includegraphics[width=10cm,height=10cm,keepaspectratio]{figures/exp_f15_k_density.pdf}
    \caption{Exponential commitment function, ftime = 15. a density plot of the aggression parameter k}
    \label{fig:k density}

    \includegraphics[width=10cm,height=10cm,keepaspectratio]{figures/exp_f15_e_0_density.pdf}
    \caption{Exponential commitment function, ftime = 15. a density plot of the aggression parameter $e_0$.}
    \label{fig:e_0 density}
\end{figure}

\begin{figure}[h!]
    \centering
    \includegraphics[width=10cm,height=10cm,keepaspectratio]{figures/exp_f15_k_history.pdf}
    \caption{Exponential commitment function, ftime = 15. $k$ the grey bounds are the standard deviation}

    \includegraphics[width=10cm,height=10cm,keepaspectratio]{figures/exp_f15_e_0_history.pdf}
    \caption{Exponential commitment function, ftime = 15. $e_0$ the grey bounds are the standard deviation}
\end{figure}


%% WINNERS

\begin{figure}[h!]
    \centering
    \includegraphics[width=15cm,height=10cm,keepaspectratio]{figures/exp_f10_num_winners.pdf}
    \caption{exp ftime = 10, exp commitment Number of successful males at different patch areas}
    \label{fig:num winners expf10}

    \includegraphics[width=15cm,height=10cm,keepaspectratio]{figures/exp_f15_num_winners.pdf}
    \caption{exp ftime = 15, exp commitment. Number of successful males at different patch areas, gaps are extinctions}
    \label{fig:num winners}
\end{figure}

\clearpage
% section results (end)results


I am not sure if this is the best way to do this, the tables in the plant paper are more aesthetic.\\

\begin{table}[h!]
    \caption{Variable names and definitions in the model.}
    \centering
    \begin{tabular}{c | c | l }
        \hline
        Symbol & unit & Description\\
        \hline
        \hline
        $K$ & & number of individual males \\
        $N$ & & number of nests \\
        $R_i$ & & reproductive value of nest of nest $i$\\
        $\bar{R}$ & & average reproductive value of nest\\
        \hline
        $t_m$ & & the time at which the females begin to mature\\
        $S(t)$ = exp($-k_it$) & & The probability of an individual surviving to time t.\\
        \hline
        $m$ & $g$ & mass\\
        $m_mat$ & $g$ & The size at which 50\% of the population is mature.\\
        $E$ & kJ & kiloJoules\\
        \hline
    \end{tabular}
\end{table}

\section{Figures}

\begin{figure}[h!]
\centering
\includegraphics[width=15cm,height=15cm,keepaspectratio]{figures/area_searched}
\caption{the area searched by a male in interval $dt$}
\label{fig:area searched figure}
\end{figure}
\clearpage

\bibliography{refs}

\end{document}

